ترتیب اولیه دستورات به این صورت است:

\setLTR
\begin{lstlisting}
I1: lw R1,0(R2)   ; R1 ← Memory[R2]
I2: addi R1,R1,1  ; R1 ← R1+1
I3: sw R1,0(R2)   ; Memory[R2] ← R1
I4: addi R2,R2,8  ; R2 ← R2+8
I5: addi R4,R4,-1 ; R4 ← R4-1
I6: bne R4,R0,I1  ; branch if R4!=0
\end{lstlisting}
\setRTL

\subsubsection*{الف}
باید جایگاه
$I_5$
را طوری تغییر بدهیم که حداقل 2 مرحله زودتر از 
$I_6$
اجرا شود.

\setLTR
\begin{lstlisting}
I1: lw R1,0(R2)   ; R1 ← Memory[R2]
I2: addi R1,R1,1  ; R1 ← R1+1
I5: addi R4,R4,-1 ; R4 ← R4-1
I3: sw R1,0(R2)   ; Memory[R2] ← R1
I4: addi R2,R2,8  ; R2 ← R2+8
I6: bne R4,R0,I1  ; branch if R4!=0
\end{lstlisting}
\setRTL

\subsubsection*{ب}
در این بخش ما دستوری را بعد از 
$I_6$
قرار می‌دهیم که هر بار اجرا شود و ارتباطی با شرط پرش نداشته باشد.

\setLTR
\begin{lstlisting}
I1: lw R1,0(R2)   ; R1 ← Memory[R2]
I2: addi R1,R1,1  ; R1 ← R1+1
I5: addi R4,R4,-1 ; R4 ← R4-1
I3: sw R1,0(R2)   ; Memory[R2] ← R1
I6: bne R4,R0,I1  ; branch if R4!=0
I4: addi R2,R2,8  ; R2 ← R2+8
\end{lstlisting}
\setRTL

\subsubsection*{ج}

با توجه به خواسته سوال، جدول زیر را تشکیل می‌دهیم:
\setLTR

$ \ \ \includegraphics[width=1\linewidth]{figs/1.png}$

\setRTL

اگر فقط یک بار این حلقه اجرا شود، در مجموع 10 کلاک زمان می‌برد، اما اگر چندین بار این حلقه تکرار شود، تقریبا به تعداد حلقه‌ها نیاز به کلاک داریم.



