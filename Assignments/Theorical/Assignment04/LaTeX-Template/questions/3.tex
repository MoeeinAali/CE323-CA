در این سوال باید توضیح دهیم که با گیر کردن سیگنال، اجرای کدام دستورات دچار مشکل خواهند شد.


\subsubsection*{الف}
اگر بیت RegDst روی صفر گیر کند، آنگاه رجیستر مقصد هم روی rt گیر می‌کند. در این حالت فقط دستوراتی دچار مشکل می‌شوند که نیاز دارند چیزی را روی rd بنویسیند، پس فقط \textbf{Type R} دچار مشکل می‌شوند.
\subsubsection*{ب}
اگر بیت صفرم ALUOp روی مقدار یک گیر کند، دستورات \textbf{lw و sw و Type R} که در آن‌ها بیت صفرم ALUOp برابر صفر است دچار مشکل می‌شوند.

البته اگر در دستورات Type R بیت صفرم ALUOp را روی حالت Care Dont در نظر گرفته باشیم، برای این عملیات‌ها مشکلی به وجود نخواهد آمد، زیرا مقدار 11 و 10 برای ما فرقی ندارد.
\subsubsection*{ج}
اگر سیگنال RegWrite روی صفر گیر کند، آنگاه روی هیچ رجیستری چیزی نوشته نمی‌شود و درنتیجه دستوراتی مانند \textbf{Type R و lw} به مشکل می‌خورند. به عنوان مثال در دستورات Type R باید نتیجه عملیات را روی یک رجیستر مقصد نوشت، که این کار انجام نخواهد شد، و یا در دستور lw باید مقداری از حافظه بخوانیم و در یک رجیستر بنویسیم که این کار انجام نخواهد شد.

همچنین در بعضی از دستورات Immediate که نوشتن روی رجیستر و یا حافظه داریم، به مشکل خواهیم خورد.