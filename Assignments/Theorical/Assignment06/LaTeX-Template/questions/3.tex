در ابتدا تعداد بلوک‌ها را بدست می‌آوریم:

\setLTR
$
\frac{1MB}{16Byte} = \frac{2^{20}}{16\times8} = 2^{13} 
$
\setRTL

طول هر بلوک 16 بایت و 4 بیت مربوط به offset و 10 بیت برای برچسب‌ها داریم. پس تعداد بیت مربوط به index برابر است با:

\setLTR
$
24-4-10 = 10bit
$
\setRTL

چون ما $2^{13}$ بلوک و $2^{10}$ مجموعه داریم، پس درنتیجه 8 بلوک در هر مجموعه داریم. درنتیجه تعداد بیت‌های لازم در هر بلوک برابر است با:

\setLTR
$
16\times8+10+1 = 139 bits
$
\setRTL

پس تعداد کل بیت‌ها برابر است با:

\setLTR
$
n = 2^{13} \times139 = 1138688bits
$
\setRTL

اگر این حافظه associative fully باشد، بیت index نداریم و 20 بیت برای برچسب خواهیم داشت. حال دوباره تعداد بیت‌های هر بلوک را محاسبه می‌کنیم:

\setLTR
$
16\times8 + 20 + 1 =149bits \longrightarrow n = 2^{13} \times 149 = 1220608bits
$
\setRTL

اگر این حافظه mapping direct باشد، 13 بیت index داریم و 7 بیت برای برچسب خواهیم داشت. حال دوباره تعداد بیت‌های هر بلوک را محاسبه می‌کنیم:

\setLTR
$
16\times8 + 7 + 1 =136bits \longrightarrow n = 2^{13} \times 136 = 1114112bits
$
\setRTL
















