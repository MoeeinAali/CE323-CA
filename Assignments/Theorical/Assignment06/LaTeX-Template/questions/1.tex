ابتدا در هر دو حالت AMAT را حساب می‌کنیم:

\setLTR
$
AMAT_1 = ht + mt_1 \times t \\
AMAT_2 = ht + mt_1 \times \frac{t}{2}  + mt_2 \times t
$
\setRTL

می‌دانیم که باید 
$AMAT_2 < AMAT_1$
باشد، پس:

\setLTR
$
AMAT_2 < AMAT_1 \longrightarrow ht + mt_1 \times \frac{t}{2} + mt_2 \times t < ht + mt_1 \times t \longrightarrow mt_2\times t < mt_1 \times  \frac{t}{2} \longrightarrow mt_2 < \frac{mt_1}{2} 
$
\setRTL

